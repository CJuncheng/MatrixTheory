\newpage
\section{线性代数基础}
\subsection{线性空间与子空间}

\begin{definition}[线性空间]
	设$V$是一个非空集合,$P$是一个数域.在集合$V$中定义加法运算:$\forall \alpha,\beta \in V$,$\exists$唯一的$v=\alpha+\beta\in V$;在集合$V$中定义数乘运算:$\forall \alpha \in V, \forall k \in P $,$\exists$唯一的$\delta=k\alpha\in V$.如果加法和数乘满足下述法则:
	\begin{enumerate}
		\item $\alpha+\beta=\beta+\alpha$
		\item $(\alpha+\beta)+v=\alpha+(\beta+v)$
		\item  $\exists\ 0 \in V, \forall \ \alpha \in V $, 有 $ \alpha+0=\alpha $
		\item $\forall \alpha \in V, \exists \beta \in V $, $s.t.\ \alpha+\beta=0 $
		\item $1\alpha=\alpha$
		\item $k(l \alpha)=kl( \alpha)$
		\item  $(k+l) \alpha=k \alpha+l\alpha$
		\item  $k(\alpha+\beta)=k \alpha+k\beta$
	\end{enumerate}
则$V$称为数域$P$上的一个线性子空间.
\end{definition}

\begin{note}
	\textcolor{red}{线性空间同时满足加法封闭和数乘封闭.}
\end{note}

\begin{definition}[线性空间的基和维数]
在$V$ 中有$n$个线性无关的向量$\varepsilon_1,\varepsilon_2,\cdots,\varepsilon_n $
而$V$ 中任意$n+1$个向量都线性相关,则称$\varepsilon_1,\varepsilon_2,\cdots,\varepsilon_n $是 $V$的一
组基,$n$就是线性空间的维数.常记为$\mathrm{dim}V=n.$
\end{definition}

\begin{definition}[线性子空间]
如果数域$P$上的线性空间$V$的一非空子集$W$
	对于$V$的两种运算也构成线性空间,则称$W$
	是$V$ 的线性子空间(简称子空间). 
\end{definition}
\begin{note}
	\textcolor{red}{平凡空间就是都是零向量的空间,一个线性空间里面只要有一个非零向量,就是非平凡空间.}
\end{note}

\subsection{空间分解与维数定理}

\begin{definition}[和空间]
	设$V_1$和$V_2$是线性空间$V$的子空间,则$V_1$与$V_2$的和空间记为
	\[
	V_1+V_2=\{\alpha_1+\alpha_2|\forall\alpha_1\in V_1, \forall\alpha_2\in V_2\}
	\]
\end{definition}

\begin{note}
	\textcolor{red}{$V_1+V_2$是包含$V_1$和$V_2$的最小子空间.}
\end{note}

\begin{definition}[交空间]
	设$V_1$和$V_2$是线性空间$V$的子空间,则$V_1$与$V_2$的交空间记为
	\[
	V_1\cap V_2=\{\alpha|\alpha\in V_1\mbox{且} \alpha\in V_2\}
	\]
\end{definition}

\begin{note}
	\textcolor{red}{$V_1\cap V_2$为包含于$V_1$与$V_2$的最大子空间.}
\end{note}

\begin{definition}[生成子空间]
	$\alpha_1,\alpha_2,\cdots,\alpha_n \in V$, 则生成子空间$W$记为
	\[
	\begin{split}
		W&=\{\beta|\beta=\sum_{i=1}^{r}k_i\alpha_i\}=\mathrm{span}\{\alpha_1,\alpha_2,\cdots,\alpha_n\}\\
		&=L(\alpha_1,\alpha_2,\cdots,\alpha_n)
	\end{split}
	\]
\end{definition}

\begin{theorem}
设$V_1$和$V_2$是线性空间$V$的子空间,则

\centering\colorbox{yellow}{$
\mathrm{dim}(V_1)+\mathrm{dim}(V_1)=\mathrm{dim}(V_1+V_2)+\mathrm{dim}(V_1\cap V_2)$}
\end{theorem}


\begin{note}
	\textcolor{red}{子空间和的维数一般比子空间的维数之和小.}
\end{note}

\begin{definition}
	设$V_1$和$V_2$是线性空间$V$的子空间,若对$\forall \alpha\in V_1+V_2$,有$\alpha=\alpha_1+\alpha_2(\alpha_1 \in V_1, \alpha_2 \in V_2)$且是唯一的,这个和$V_1+V_2$就称为直和,记为$V_1\oplus V_2.$
\end{definition}

\begin{theorem}
	设$V_1$和$V_2$是线性空间$V$的子空间,则下列命题等价
		\begin{enumerate}
		\item $V_1+V_2$是直和
		\item 零向量表示法唯一
		\item  $V_1\cap V_2=\{ 0\} $
		\item  $ \mathrm{dim}(V_1\cap V_2)=0 \Leftrightarrow  \mathrm{dim}(V_1)+\mathrm{dim}(V_1)=\mathrm{dim}V_1+V_2)$
	\end{enumerate}
\end{theorem}


\begin{definition}
	设$V_1,V_2,\cdots, V_s$线性空间$V$的子空间,如果和$V_1+V_2+\cdots+ V_s$
	中的每个向量$\alpha$的分解式
	\[
	\alpha=\alpha_1+\alpha_2+\cdots+\alpha_s, \alpha_i \in V_i(i=1,2,\dots, s)
	\]
	是唯一的,这个和$V_1,V_2,\cdots, V_s$就称为直和,记为$V_1\oplus V_2\oplus\dots \oplus V_s.$
\end{definition}


\begin{theorem}
设$V_1,V_2,\cdots, V_s$线性空间$V$的子空间,则下列命题等价
	\begin{enumerate}
		\item $W=V_1+V_2+\cdots+ V_s$是直和
		\item 零向量表示法唯一
		\item  $V_i\cap\sum\limits_{(j\ne i)} V_j=\{ 0\} $
		\item  $ \mathrm{dim}(W)=\sum \mathrm{dim}(V_i)$
	\end{enumerate}
\end{theorem}

\subsection{特征值与特征向量}
\subsubsection{特征值和特征向量的概念}
\begin{definition}
$A\in C^{n\times n}$,如果存在$\lambda \in C$和非零向量$x \in C^n$,使得
\[
Ax=\lambda x
\]
则$\lambda$叫$A$的特征值,$x$叫$A$的属于特征值$\lambda$的特征向量.

\end{definition}
\begin{note}
\begin{enumerate}
	\item \textcolor{ecolor}{矩阵的普:}$A$的所有特征值全体,叫做$A$的普,记为$\lambda(A).$
	\item \textcolor{ecolor}{特征多项式:}
	\[
	|\lambda E-A|=\left(\lambda-\lambda_{1}\right)^{n_{1}} \cdots\left(\lambda-\lambda_{i}\right)^{n_{i}} \cdots\left(\lambda-\lambda_{r}\right)^{n_{r}}
	 \left(\mbox{其中}\sum_{i=1}^{r} n_{i}=n_{0}\right).
	\]
	\item  \textcolor{ecolor}{代数重数:}$n_i$叫做$\lambda_i$的代数重数.(\textcolor{red}{特征值重数.})
	\item  \textcolor{ecolor}{几何重数:}$W=\{x|(\lambda_iE-A)x=0\}, m_i=\mathrm{dim}(W)=n-rank(\lambda_iE-A)$叫做$\lambda_i$的几何重数.(\textcolor{red}{特征值$\lambda_i$对应的线性无关的特征向量的个数,至多有$n_i$个.})
	\item \colorbox{yellow}{$m_i<n_i$}
	\item  \textcolor{ecolor}{不同特征值对应的特征向量线性无关.}
\end{enumerate}
\end{note}

\begin{definition}
	设$A\in C^{n\times n}$,如果存在可逆矩阵$P\in C^{n\times n}$,使得
	\[
	P^{-1}AP=diag(\lambda_1,\lambda_2,\cdots,\lambda_n)
	\]
	则称矩阵$A$叫做可对角化矩阵.
	
\end{definition}

\begin{note}
	\textcolor{red}{$n$阶矩阵$A$可对角化$\Leftrightarrow$ $n$阶矩阵$A$有$n$个线性无关的特征向量.}
\end{note}

\begin{theorem}
	设$A\in C^{n\times n}$,则存在可逆矩阵$P\in C^{n\times n}$,使得
	\[
	P^{-1}AP=J=diag\left(J(\lambda_1),J(\lambda_2),\cdots,J(\lambda_r)\right)
	\]
	则称$J$为$A$的Jordan标准型.
\end{theorem}
\begin{note}
	
\begin{enumerate}
	\item Jordan块的个数$r$是线性无关特征向量的个数,则$r=n(m_i=n_i)\Leftrightarrow 
	$矩阵$A$可对角化
	\item 由于同一特征值可以对应多个线性无关特征向量,所以$\lambda_1,\lambda_2,\cdots,\lambda_r$不一定不相同.
	\item  对应于同一特征值的Jordan块的个数是该特征值
	的几何重数, 它是相应的特征子空间的维数.
	\item  对应于同一特征值的所有Jordan块的阶数之和是
	该特征值的代数重数.
\end{enumerate}
\end{note}

\begin{theorem}
	设$A\in C^{n\times n}$,则下列命题等价:
	\begin{enumerate}
		\item 矩阵$A$可对角化
		\item $C^{n}$存在由$A$的特征值向量构成的一组基底.
		\item  $A$的Jordan标准型中的Jordan块都是一阶的.
		\item  $m_i=n_i(i=1,2,\cdots,r)$
	\end{enumerate}
\end{theorem}

\subsubsection{特征值和特征向量的几何性质}
\begin{enumerate}
	\item \textcolor{ecolor}{变换}
	\[
	  \begin{split}
	    &V\stackrel{T}{\longrightarrow}V	\\
	    &\forall\alpha\in V,\alpha \stackrel{T}{\longrightarrow}\alpha^{'} \in V
	 \end{split}
	\]
	(线性空间$V$中的任一元素$\alpha$,都有$V$中唯一确定的元素$\alpha^{'}$与之对应)则称$T$为$V$的变换.
	\item \textcolor{ecolor}{线性变换}
	
	$T$为$V$的变换且满足
	\[
	\left\{
	\begin{array}{l}
		\forall \alpha,\beta \in V, T(\alpha+\beta)= T(\alpha)+T(\beta)\\
	\forall k \in P, T(k\alpha)=kT(\alpha)
	\end{array} \right.
	\]
	则称$T$为$V$的线性变换.

	\item  \textcolor{ecolor}{线性变换的特征值}
	\begin{definition}
		设$T$ 是线性空间$V_n(C)$的一个线性变换,如果存在$\lambda\in C$和非零向量$\xi\in V_n(C)$,使得$T\xi=\lambda\xi$,则$\lambda$叫$T$的特征值,$\xi$叫$T$的属于特征值$\lambda$的特征向量.
	\end{definition}
	
	\item \textcolor{ecolor}{线性变换与矩阵}
	\[
	\begin{split}
		T(\epsilon_1,\epsilon_2,\cdots, \epsilon_n)&=	(T\epsilon_1,T\epsilon_2,\cdots, T\epsilon_n)
		=(\epsilon_1,\epsilon_2,\cdots, \epsilon_n)
		\begin{pmatrix}
			a_{11}&a_{12}&\cdots&a_{1n}\\ 
			a_{21}&a_{22}&\cdots&a_{2n}\\ 
			\vdots&\vdots&&\vdots\\ 
			a_{n1}&a_{n2}&\cdots&a_{nn}
	   \end{pmatrix}\\
   &=(\epsilon_1,\epsilon_2,\cdots, \epsilon_n)A
	\end{split}
	\]
	$A$是线性变换$T$在基$\epsilon_1,\epsilon_2,\cdots, \epsilon_n$下的矩阵.
	\item \textcolor{ecolor}{线性变换与矩阵特征值关系}
	\[
	\begin{split}
	&\alpha=\sum\limits_{i=1}^{n}x_i\epsilon
	_i,T\alpha=\lambda\alpha\\
	&\Rightarrow T\alpha=(T\epsilon_1,T\epsilon_2,\cdots, T\epsilon_n)x=(\epsilon_1,\epsilon_2,\cdots, \epsilon_n)Ax,\lambda\alpha=(\epsilon_1,\epsilon_2,\cdots, \epsilon_n)\lambda x\\
	&\Rightarrow Ax=\lambda x
\end{split}
	\]
	   \item \textcolor{ecolor}{线性变换在不同基下矩阵之间的关系}
	\begin{theorem}
		\[
		\begin{split}
			&T\xrightarrow[]{(\epsilon_1,\epsilon_2,\cdots, \epsilon_n)}A,T\xrightarrow[]{(\epsilon_1^{'},\epsilon_2^{'},\cdots, \epsilon_n^{'})}B\\
			&(\epsilon_1^{'},\epsilon_2^{'},\cdots, \epsilon_n^{'})=(\epsilon_1,\epsilon_2,\cdots, \epsilon_n)C
		\end{split}\Rightarrow B=C^{-1}AC
		\]
	\end{theorem}
\end{enumerate}

\subsubsection{广义特征值问题}
\begin{definition}
	设$A,B\in C^{n\times n}$,如果存在$\lambda \in C$和非零向量$x \in C^n$,使得
	\[
	Ax=\lambda Bx
	\]
	则$\lambda$叫$A$与$B$确定的广义的特征值,$x$称为与$\lambda$的广义特征向量.
\end{definition}

\begin{note}
	\begin{itemize}
		\item $B=E\Rightarrow Ax=\lambda x(x\ne0)$
		\item $B$可逆,$B^{-1}Ax=\lambda x$
		\item 当$A,B$ 都是Hermite矩阵,即$A=A^H,B=B^H$且$B$正定时,有
		
		 \colorbox{yellow}{
			$
		Qy=\lambda y\left(\mbox{其中}Q=(P^{-1})^HAP^{-1},y=Px,B=P^HP     \right)
		$}
	
	\colorbox{yellow}{	($B=B^H$且$B$正定$\xrightarrow[]{\mbox{存在可逆矩阵}P}B=P^HP$
	)}
	
	由于广义特征值是实数,则$y_1,y_2,\cdots,y_n$构成标准正交基,当
	\[
	\delta_{ij}=y_i^Hy_j=(Px_i)^H(Px_j)=x_i^HP^HPx_j=x_i^HBx_j=\left\{
	\begin{array}{ll}
		1&,i=j\\
		0&,i\ne j
	\end{array} \right.
	\],称$x_1,x_2,\cdots,x_n$为$B$的共轭向量系.
	\end{itemize}
\end{note}


\begin{theorem}
	设 \( n \times n \) 矩阵 \( A=A^{H}, B=B^{H} \), 且 \( B \) 正定, 则 \( B \) 共轭 向量系 \( x_{1}, x_{2}, \cdots, x_{n} \) 具有以下性质:
	
		\begin{itemize}
		\item \( x_{i} \neq 0(i=1,2, \cdots, n) \);
		\item \( x_{1}, x_{2}, \cdots, x_{n} \) 线性无关;
		\item \( \lambda_{i} \) 与 \( x_{i} \) 满足方程 \( A x_{i}=\lambda_{i} B x_{i} \);
		\item 若令 \( X=\left(x_{1}, x_{2}, \cdots, x_{n}\right) \),
		\( X^{H} B X=E, X^{H} A X=\operatorname{diag}\left(\lambda_{1}, \lambda_{2}, \cdots, \lambda_{n}\right) \)
	\end{itemize}
\end{theorem}

\subsection{欧氏空间和酉空间}
\subsubsection{内积空间}


\( \left\{\begin{array}{l}V(P)\cdots \cdots P \mbox{上的线性空间} \\ (\bullet,\bullet )\cdots \cdots \text { 内积(特殊的二元函数) }\end{array}\cdots \cdots V\right. \) 为 \( P \) 上的内积空间.

内积: \( V \rightarrow P \) 上的二元函数 \( ( \) 或 \( V \times V \rightarrow P): f(0,0)=(0,0) \) 满足:
\( \forall \alpha, \beta \in V, f(\alpha, \beta)=k \in P \)


\begin{itemize}
\item (双线性): \( \left\{\begin{array}{l}f\left(x, a_{1} y_{1}+a_{2} y_{2}\right)=a_{1} f\left(x, y_{1}\right)+a_{2} f\left(x, y_{2}\right) \\ f\left(a_{1} x_{1}+a_{2} x_{2}, y\right)=\bar{a}_{1} f\left(x_{1}, y\right)+\bar{a}_{2} f\left(x_{2}, y\right)\end{array}, a_{i} \in P\right. \)
\item(正定性及对称性): \( \left\{\begin{array}{lr}f(x, y)=\overline{f(y, x)}&,\mbox{对称性} \\ f(x, x) \geq 0 \text { 且 } f(x, x)=0 \Leftrightarrow x=0&,\mbox{正定性} \end{array}\right. \)
\end{itemize}

\begin{note}
\( \left\{\begin{array}{l} P=R \text { : 内积空间称为欧氏空间}  \\  P=C \text { : 内积空间称为酉空间 }\end{array}\right. \).
\end{note}

\begin{enumerate}
\item \textcolor{ecolor}{欧氏空间}
\begin{definition}
	在线性空间 \( V_{n}(R) \) 上, \( \forall \alpha, \beta, \gamma \in V \), 若映射 \( (\alpha, \beta) \) 满足
	\begin{enumerate}
	 \item(正定性): \( (\alpha, \alpha) \geq 0 ;(\alpha, \alpha)=0 \Leftrightarrow \alpha=0 \),
	\item(齐次性): \( (k \alpha, \beta)=k(\alpha, \beta) \)
	\item(交换律): \( (\alpha, \beta)=(\beta, \alpha) \)
	\item(分配律): \( (\alpha+\beta, \gamma)=(\alpha, \gamma)+(\beta, \gamma) \)
	\end{enumerate}
	则映射 \( (\alpha, \beta) \) 是 \( V_{n}(R) \) 上的内积, 定义了内积的 \( V \) 为 \( n \) 维欧几里得空间,简称欧氏空间.
\end{definition}

\item \textcolor{ecolor}{酉空间}	
\begin{definition}
	在线性空间 \( V_{n}(R) \) 上, \( \forall \alpha, \beta, \gamma \in V \), 若映射 \( (\alpha, \beta) \) 满足
	\begin{enumerate}
		\item(正定性) \( (\alpha, \alpha) \geq 0 ;(\alpha, \alpha)=0 \Leftrightarrow \alpha=0 \),
		\item(齐次性) \( (k \alpha, \beta)=\bar{k}(\alpha, \beta) \)
		\item(交换律): \( (\alpha, \beta)=\overline{(\beta, \alpha)} \)
		\item(分配律): \( (\alpha+\beta, \gamma)=(\alpha, \gamma)+(\beta, \gamma) \)
	\end{enumerate}
	则映射 \( (\alpha, \beta) \) 是 \( V_{n}(R) \) 上的内积, 定义了内积的 \( V \) 为 \( n \) 维酉空间,简称欧氏空间.
\end{definition}

\end{enumerate}


\subsubsection{欧氏(酉)空间的度量}

\subsubsection{内积的应用}
\subsubsection{补充:初等矩阵}

